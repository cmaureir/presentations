\frame
{
\frametitle{}
\begin{center}
	\Huge{Historia}
\end{center}
}

\frame
{
\frametitle{Historia}
\begin{itemize}
	\item Linux hace su \textbf{aparición} a principios de la década de los 90.
	\item Un \textbf{estudiante} de informática de la Universidad de Helsinki
		llamado \emph{Linus Torvalds}, lo comenzó como un hobby.
	\item Linus nunca pensó que tanta gente se interesara en ayudarlo
		ni tampoco en lo grande que se convertiría su proyecto.
\end{itemize}
}
\frame
{
\frametitle{Historia}
\begin{itemize}
	\item Su creación estuvo \textbf{inspirado} en MINIX, un pequeño sistema Unix
		desarrollado por Andy Tanenbaum.
	\item Las primeras \textbf{discuciones} de Linux fueron por una lista de correos
		donde Linus pedía consejos y feedback.
\end{itemize}
}

\frame
{
\frametitle{Historia}
\begin{block}{Email}
	Hello everybody out there using minix -

	I'm doing a (free) operating system (just a hobby, won't be big and
	professional like gnu) for 386(486) AT clones.$\ldots$\\\vspace{0.5cm}
	Any suggestions are welcome, but I won't promise I'll implement them $:-)$\\\vspace{0.5cm}
	$\ldots$
	PS.  Yes - it's free of any minix code, and it has a multi-threaded fs. 
	It is NOT protable (uses 386 task switching etc), and it probably never
	will support anything other than AT-harddisks, as that's all I have :-(. 
\end{block}
}
